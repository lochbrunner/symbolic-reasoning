\documentclass{scrartcl}

\usepackage[utf8]{inputenc}
\usepackage[T1]{fontenc}
\usepackage{lmodern}
\usepackage{babel}
\usepackage{amsmath}
\usepackage{mathtools}
\usepackage{xcolor}

\usepackage[
backend=biber,
% style=alphabetic,
sorting=none
]{biblatex}

\addbibresource{whitepaper.bib}

\title{Rule based Symbolic Reasoning}
\author{Matthias Lochbrunner}
\date{\today}

\begin{document}
\maketitle
\begin{center}
	\textbf{Abstract}
\end{center}
\begin{abstract}
	Solving mathematical problems with a rule based beam search using deep neuronal networks for policy and value estimation.
	This natural approach is capable of solving problems in formal science. 
\end{abstract}

\section{Motivation}
\begin{itemize}
	\item Adapt human like approach and how they learn.
	\item Scalable: This solution can be applied on each formal problem of any domain.
	\item Correctness: The result is always correct as the generation process is based on hard correct rules.
	\item Applicable to literature: As the network can grow it can incorporate new information formatted as rules as humans do. 
\end{itemize}

\section{Method}

\subsection{Fit and Apply}

\textbf{Example}

Consider the derivative rule
\begin{align}
	\frac{\partial}{\partial {\color{red}x}} {\color{red}x}^{\color{blue}n} \Longrightarrow n\cdot {\color{red}x}^{{\color{blue}n}-1} 
\end{align}
applied to the term
\begin{align}
	\frac{\partial}{\partial {\color{red}z}} {\color{red}z}^4 
\end{align}.
Using the mapping ${\color{red}x \coloneqq z}$ and ${\color{blue}n \coloneqq {\color{blue}4}}$ leads to the new term
\begin{align}
	{\color{blue}4}\cdot {\color{red}z}^{{\color{blue}4}-1} 
\end{align}

\subsection{Policy and Value Networks}

Based on two different architecture multi-head networks for value and policy estimation \cite{44806}.

The policy head suggests useful locations and rules to apply on the given term. This outcome is used to shrink the width of the search beam.
It turned out that is very important to suppress non-sense rule application as they can blow up the beam. (Negative training samples.)

The value head estimates how likely the given term might lead to success. If the value is below a specified threshold the track gets cancelled.

\subsubsection{Challenges for Tree Structured Data}
As each term is represented as directed async graph standard neuronal operation can not be performed on this input.
The CNN based network uses an adapted operation for tree structured data.
Special positional encoding for transformer based architecture.

\subsection{Data loop}

From given (training) problems the algorithm trains itself by the three-T-loop: 

\begin{itemize}
	\item \textbf{Try} to solve some of the training problems.
	\item \textbf{Trace} the calculation steps of the solved problems and create training samples out of them.
	\item \textbf{Train} the model with these useful and useless trainings steps created.
\end{itemize}

\section{Results}
We have defined all basic rules to perform the given problems as scenario configuration.

% \subsection{Scenario Definition}

\section{Conclusion}

\section{Outlook}
\begin{itemize}
	\item Support propositional logic to increase the set of mathematical problem
	\item Insert new rules (macro rules) into the existing network. (Growable tree structured network architecture). \cite{graves2016hybrid}
\end{itemize}

\section{Related work}

\begin{itemize}
	\item Using state of the art sequence to sequence transformers on polish notation formatted equations \cite{Lample2020Deep}. Differences: 1D data instead of Tree structured data. Not based on rules. 
\end{itemize}

\printbibliography
\end{document}

