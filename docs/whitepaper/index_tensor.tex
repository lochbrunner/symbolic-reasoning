
    \tikzset{
        hatch distance/.store in=\hatchdistance,
        hatch distance=24pt,
        hatch thickness/.store in=\hatchthickness,
        hatch thickness=8pt
    }
\makeatletter
\pgfdeclarepatternformonly[\hatchdistance,\hatchthickness]{flexible hatch}
{\pgfqpoint{0pt}{0pt}}
{\pgfqpoint{\hatchdistance}{\hatchdistance}}
{\pgfpoint{\hatchdistance-1pt}{\hatchdistance-1pt}}%
{
    \pgfsetcolor{\tikz@pattern@color}
    \pgfsetlinewidth{\hatchthickness}
    \pgfpathmoveto{\pgfqpoint{0pt}{0pt}}
    \pgfpathlineto{\pgfqpoint{\hatchdistance}{\hatchdistance}}
    \pgfusepath{stroke}
}
\makeatother

\subfigure[Neighbors of the root.
$\scriptstyle \phi'_=={\color{red}w_1\phi_=}+{\color{blue}w_2\phi_+}+{\color{green!70!black}w_3\phi_-}+{\color{orange!80!black}w_4\phi_{\text{PAD}}}$]  
{
\tikzset{g_style/.style={black, draw=black!15, minimum size=0.5cm}}
\begin{tikzpicture}
    \node[g_style,fill=red!20,draw=black] (=) at (0,0){=};
    \node[g_style,fill=blue!20] (+) at (-1,-1){$+$};
    \node[g_style,fill=green!20] (-) at (1,-1){$-$};
    \node[g_style] (a) at (-1.5, -2) {$a$};
    \node[g_style] (b) at (-0.5, -2) {$b$};
    \node[g_style] (c) at (0.5, -2) {$c$};
    \node[g_style] (d) at (1.5, -2) {$d$};
    \node[g_style,fill=orange!20] at (1.5, 0) {\tiny PAD};

    \draw[draw=blue!70] (=) -- (+);
    \draw[draw=green!70] (=) -- (-);
    \draw[] (+) -- (a);
    \draw[] (+) -- (b);
    \draw[] (-) -- (c);
    \draw[] (-) -- (d);
\end{tikzpicture}
}
\subfigure[Neighbors of a left node $+$
$\scriptstyle \phi'_+={\color{red}w_1\phi_+}+{\color{blue}w_2\phi_a}+{\color{green!70!black}w_3\phi_b}+{\color{orange!80!black}w_4\phi_=}$]  
{
\tikzset{g_style/.style={black, draw=black!15, minimum size=0.5cm}}
\begin{tikzpicture}
    \node[g_style,fill=orange!20] (=) at (0,0){=};
    \node[g_style,fill=red!20,draw=black] (+) at (-1,-1){$+$};
    \node[g_style] (-) at (1,-1){$-$};
    \node[g_style,fill=blue!20] (a) at (-1.5, -2) {$a$};
    \node[g_style,fill=green!20] (b) at (-0.5, -2) {$b$};
    \node[g_style] (c) at (0.5, -2) {$c$};
    \node[g_style] (d) at (1.5, -2) {$d$};
    \node[g_style] at (1.5, 0) {\tiny PAD};

    \draw[draw=orange!70] (=) -- (+);
    \draw[] (=) -- (-);
    \draw[draw=blue!70] (+) -- (a);
    \draw[draw=green!70] (+) -- (b);
    \draw[] (-) -- (c);
    \draw[] (-) -- (d);
\end{tikzpicture}
}
\subfigure[Neighbors of a leaf $b$
$\scriptstyle \phi'_b={\color{red}w_1\phi_b}+{\color{blue}w_2\phi_{\text{PAD}}}+{\color{green!70!black}w_3\phi_{\text{PAD}}}+{\color{orange!80!black}w_4\phi_+}$]  
{
\tikzset{g_style/.style={black, draw=black!15, minimum size=0.5cm}}
\begin{tikzpicture}
    \node[g_style] (=) at (0,0){=};
    \node[g_style,fill=orange!20] (+) at (-1,-1){$+$};
    \node[g_style] (-) at (1,-1){$-$};
    \node[g_style] (a) at (-1.5, -2) {$a$};
    \node[g_style,fill=red!20,draw=black] (b) at (-0.5, -2) {$b$};
    \node[g_style] (c) at (0.5, -2) {$c$};
    \node[g_style] (d) at (1.5, -2) {$d$};
    {
        \node[preaction={fill=blue!20}, g_style, pattern=flexible hatch, pattern color=green!20] at (1.5, 0) {\tiny PAD};
    }

    \draw[] (=) -- (+);
    \draw[] (=) -- (-);
    \draw[] (+) -- (a);
    \draw[draw=orange!70] (+) -- (b);
    \draw[] (-) -- (c);
    \draw[] (-) -- (d);
\end{tikzpicture}
}
\caption{
    Example of indexed convolution operation on term $a+b=c+d$.
    Sub-figure (b) shows the computation of the next value $\phi' = C\left(\phi\right)$ for the node $a$ in that term.
    It is the weighted sum of the current values of itself, it's children and the parent in that order.
    As the root in (a) has no parent node, we use the padding as a default node.
    The same is true for the child-nodes of a leaf in sub-figure (c).
}